\documentclass[draft]{rureport}
\svnid{$Id: example-proposal-detailed.tex 145 2016-08-16 11:56:53Z foley $}
%% Reykjavik University Proposal Template (detailed)
%% Written by Joseph Timothy Foley <foley AT ru DOT IS>

%% Fixmes are reminders of things that still need to be done.
%% The default fixmes are:  \fxnote{} \fxwarning{} \fxerror{} \fxfatal{} (same as \fixme{})
% if you want personalized fixmes, then register the authors here
% notice that the first field is 2 letters, the second is 3.
\FXRegisterAuthor{jf}{jtf}{foley}
% this registers \jfnote{}, \jfwarning{}, \jferror{}, \jffatal{}

%% declare the paths(s) where you graphics files can be found
\graphicspath{{graphics/}{Graphics/}{./}}

\author{
  Ardís\formatemail{ardis@ru.is}
  \and
  Davíð Óttarsson\formatemail{davido22@ru.is}
  \and 
  Gunnar\formatemail{gunnar@ru.is}
  \and
  Hjördís\formatemail{hjördís@ru.is}
  \and
  Torfi\formatemail{torfi@ru.is}
}

\title{Detailed Proposal Template}  % The title, for the titlepage

\course{T-411-MECH Mechatronics 1}


\usepackage{hyperref} % must be last package loaded!
% it makes hyper-references (citations, URLs, etc) clickable
\begin{document} % this tells the compiler that it is time to make
                 % text to print instead of just getting ready.
\maketitle  % make a title page from the Title, Date, and Author
\listoffixmes{}

\section*{\LaTeX{} Hints}
\begin{itemize}
\item Put one sentence per line
\item Compile the document often and look for errors.
  If you find one, try commenting out the area to locate the source of the problem.
\item Watch out for \& and \%.  They have to have a left-slash in front of them.
\end{itemize}

\section{Introduction}
What is the idea?  What is it called and why?
Who is the target customer?

\subsection{Customer Needs}
What would a customer need the item to do?  
Using Axiomatic Design theory, this is stated as a numbered list of Customer Needs(CN)~\cite{suh1990principles}.
The top level is \CN0.
This is often (but not always) decomposed into \CN1, \CN2, etc.
Here is an example of a top level:

\begin{quote} \textbf{\CN0} A transfer bin for whole salmon, compatible with the SureTrack grader, cheaper and less prone to cracking due to skewing.  
The bin should be adaptable to a pure transfer task and be able to discharge anywhere along its path without
accidental discharge.~\cite{gerhard2016suretrack}
\end{quote}


\section{Prior Art}
What exists that is similar?  How is yours better/distinctive?
Give at least two examples and quantify the differences (numeric values).
If you say something is cheaper, you need to give the costs for both items.

\subsection*{Sources}
You will want to cite all these similar concepts/products.
As an example of a citation, Carryer et al.~\cite{carryer2011IntroMechatronics} is the textbook for T-411-MECH Mechatronics 1.


\section{Design}
As previously mentioned, using Axiomatic Design Theory is a good way to develop your design.

Here is a brief synopsis from Omarsdóttir et al.\cite{omarsdottir2016chessmate}:
\begin{quotation}
  Rather, the focus was placed on developing comprehensive FR and DP lists, then evaluating the coupling between them.
  This coupling is symbolized in a design matrix, which is a Cartesian product of all FR and DP combinations~\cite{cochran2016msdd, benevides2012aed}.
Where there is an interaction between an FR and DP, this is denoted by a non-zero coefficient, or in the case of the value being unknown, simply a placeholder variable $X$.
Minor levels of coupling, often considered higher-order effects, are annotated with $x$ to show their lessened effect.
A diagonal matrix is ``uncoupled'' and satisfies the Independence Axiom: ``to maintain the independence of the functional requirements~(FRs)''~\cite{suh2001axiomatic}.
Such a design can be easily optimized by adjusting a particular FR or DPs without affecting others.
A diagonal matrix indicates a ``decoupled'' or ``path-dependent'' solution, which can still be optimized, but the ordering of parameter choice selection becomes important.
All other design matrices are ``coupled'' and may have a usable local solution but usually resist modification and optimization~\cite{suh2001axiomatic}.
Needless to say, the focus is on minimizing coupling wherever it may appear.

ADT's second axiom is ``minimize the information content of the design.''
Simply put, ensure that the design has the highest probability of meeting the stated FRs.
When systems are not able to meet FRs all of the time, this is denoted in ADT as ``complexity'' and is deeply explored in~\cite{suh2005complexity}.
As will become apparent in the next section, this axiom became integral to the design of the interaction between the robot and its chess pieces.
Finally, any factors to be considered that are not functional are categorized as ``Constraints.''
These are often resource-focused and affect all of the design decisions; they need to be revisited often especially when choosing between otherwise equivalent implementations.
\end{quotation}
The first axiom is often called the Independence Axiom, and the second, the Information Axiom.


From the Customer Needs, we build a list of Functional Requirements.

Again, we start with a top-level \FR0: ``Contain \SI{25}{\kilogram} of fish on SureTrack conveyor until release is triggered''
From this, a top-level Design Parameter \DP0: Gable-reinforced stainless-steel locking bin with bi-directional discharge
\cite{gerhard2016suretrack}.

We continue a ``zig-zag'' procedure to decompose and map the FRs to the DPs as shown in Table~\ref{tab:first_level-frdp}.

\begin{table}
  \center
  \caption{First level FR-DP mapping.~\cite{gerhard2016suretrack}}\label{tab:first_level-frdp}
  \begin{tabular}{lll} \toprule
    ID& Functional Requirement & Design Parameter \\ \midrule 
    1&Contain product&Main weldment\\
    2&Move product&Support system\\
    3&Discharge product &Discharge system\\
    \bottomrule
  \end{tabular}
\end{table}

From this mapping we develop a design matrix as shown in Equation~\ref{eq:top-design-matrix} from~\cite{gerhard2016suretrack}.

\begin{equation}\label{eq:top-design-matrix}
\begin{Bmatrix}
\FR{1}\\
\FR{2}\\
\FR{3}
\end{Bmatrix}=
\begin{bmatrix}
X &  0 & X\\
0 &  X & 0\\
0 &  0 & X
\end{bmatrix}
\begin{Bmatrix}
\DP{1}\\
\DP{2}\\
\DP{3}
\end{Bmatrix}
\end{equation}

This matrix is de-coupled i.e. path-dependent, meaning it can be optimized, but the order matters.

\section{Schedule}
Rough schedule of how you will perform the tasks.
A per-week estimation is sufficient.
This should be in a table or list.

\begin{itemize}
\item Week 1: Brainstorm, search for parts
\item Week 2: Finish literature review and proposal
  \end{itemize}

\section{Bill of Materials}
Make a table of what you will need, quantity, vendor, and cost.
See Table~\ref{tab:bom} for an example.
You should be able to give this parts list to Hrannar (or someone else) and they will be able to purchase exactly what you want without asking questions.
At the bottom should be a total in ISK.\@
Don't forget to estimate VAT and import Tariffs.


\begin{table}
  \centering
  \caption{Bill of Materials example}\label{tab:bom}
  \begin{tabular}{lllllll}\toprule
    Vendor &Model &Description &Each &Each &Quantity &Total\\
           &      &            &{\it USD} &{\it ISK}  &         &{\it ISK}  \\\midrule
    SparkFun &123 &RedBoard &40.00 &6000 &1 &6000\\
    \midrule
    & & & & &Total &6000\\
                                              
  \end{tabular}
\end{table}

\section{Impact}
If successful, what impact will this design have?
How easy is it to turn into a business?
What would you expect for volume/sales?

\bibliographystyle{IEEEtran}
\bibliography{references}

\newpage
\appendix
\section{Quotes}
For large or expensive items, include a PDF that has the quote/invoice.

\section{Drawings}
If you have additional concept drawings or CAD, including from previous ideas, put them here.

\end{document}
%%%%%%%%%%%%%%%%%%%% TeXStudio Magic Comments %%%%%%%%%%%%%%%%%%%%%
%% These comments that start with "!TeX" modify the way TeXStudio works
%% For details see http://texstudio.sourceforge.net/manual/current/usermanual_en.html   Section 4.10
%%
%% What encoding is the file in?
% !TeX encoding = UTF-8
%% What language should it be spellchecked?
% !TeX spellcheck = en_US
%% What program should I compile this document with?
% !TeX program = pdflatex
%% Which program should be used for generating the bibliography?
% !TeX TXS-program:bibliography = txs:///bibtex
%% This also sets the bibliography program for TeXShop and TeXWorks
% !BIB program = bibtex

%%% Local Variables:
%%% mode: latex
%%% TeX-master: t
%%% End:

